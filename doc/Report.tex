\documentclass[12pt,a4paper]{report}
\pagestyle{plain}

\usepackage[utf8]{inputenc}
\usepackage[T2A]{fontenc}
\usepackage[english,russian]{babel}

\author{А. Григорьев}
\title{Доказательство с нулевым разглашением знания закрытого ключа RSA}

\usepackage[unicode=true]{hyperref}

\hypersetup{
    pdftitle =
        Доказательство с нулевым разглашением знания закрытого ключа RSA,
    pdfauthor =
        А. Григорьев
}

\makeatletter
\def\imod#1{\allowbreak\,({\operator@font mod}\ #1)}
\makeatother

\def \jacobi #1#2{\left(\frac{#1}{#2}\right)}

\begin{document}

\maketitle
\tableofcontents

%\chapter{Введение}

%\section{Постановка задачи}

%\section{Анализ предметной области}

\chapter{Используемые алгоритмы}

\section{Вспомогательные алгоритмы}

%\subsection{Вычисление символа Лежандра}
%
%Пусть $n$ - нечетное число, и $0 \leq a \le n$. Символ Якоби
%$\jacobi{a}{n}$ (который также будет являться символом Лежандра,
%если $n$ - простое) будем находить по алгоритму:
%
%\begin{enumerate}
%\item Если a = 0, то ответ 0.
%\item Если a = 1, то ответ 1.
%\item Запишем $a$ в виде  $a = 2^{e}a_1$, где $a_1, e \in Z_n$ и $a_1$ нечетно.
%\item Если $e$ четно, то установим $s \leftarrow 1$. 
%Иначе установим $s \leftarrow 1$ если $n \equiv 1$ или $7 \imod 8$, или
%установим $s \leftarrow -1$ если $n \equiv 3$ или $5 \imod 8$.
%\item Если $n \equiv 3 \imod 4$ и $a_1 \equiv 3 \imod 4$, то
%установим $s \leftarrow -s$.
%\item Установим $n_1 \leftarrow n \bmod{a_1}$.
%\item Если $a_1 = 1$, то ответом будет $s$; иначе ответ будет
%$s \cdot \jacobi{n_1}{a_1}$.
%\end{enumerate}

\subsection{Расширенный алгоритм Эвклида}

Пусть $a$ и $b$ -- положительные целые числа, и $a \geq b$.
Будем вычислять $d = \textrm{НОД}(a, b)$ и числа $x$, $y$ удовлетворяющие равенству $ax + by = d$ по следующему алгоритму:

\begin{enumerate}
\item Если $b = 0$, то ответом будет $d = a$, $x = 1$, $y = 0$.
\item Установим $x_2 \leftarrow 1$, $x_1 \leftarrow 0$, $y_2 \leftarrow 0$,
$y_1 \leftarrow 1$.
\item Пока $b > 0$, будем выполнять следующее:
\begin{enumerate}
\item $q \leftarrow \lfloor a / b \rfloor$, $r \leftarrow a - qb$,
$x \leftarrow x_2 - qx_1$, $y \leftarrow y_2 - qy_1$.
\item $a \leftarrow b$, $b \leftarrow r$, $x_2 \leftarrow x_1$,
$x_1 \leftarrow x$, $y_2 \leftarrow y_1$, $y_1 \leftarrow y$.
\end{enumerate}
\item Ответ: $d = a$, $x = x_2$, $y = y_2$.
\end{enumerate}

%\section{Вычисление квадратного корня в $Z_n$}
%
%Операции возведения во вторую степень по модулю $n$ и вычисления квадратных
%корней по модулю $n$ широко используются в криптографии.
%Операция вычисления квадратных корней по модулю $n$ легко выполнима при простом
%$n$, но достаточно сложна если $n$ - составное число, для которого не известно
%разложение на простые множители.

%\subsection{Вычисление квадратного корня по простому модулю}

%Если $p$ -- простое, то для любого $a \in Z^*_p$ можно легко определить,
%является ли оно квадратичным вычетом по модулю $p$, так как по определению $a
%\in Q_p$ тогда и только тогда, когда $\jacobi{a}{p} = 1$. Символ Лежандра
%$\jacobi{a}{p}$ легко вычислить по указанному выше алгоритму.\\

%Алгоритм вычисления квадратных корней числа $a$ по простому модулю $p$:

%\begin{enumerate}
%\item Вычисляем символ Лежандра. Если $\jacobi{a}{p} = -1$, то $a$ не имеет
%квадратных корней по модулю $p$.
%\item Будем перебирать целые числа $b$, $1 \leq b \leq p - 1$, случайным
%образом, пока не найдем такое, что $\jacobi{b}{p} = -1$ (то есть пока не 
%найдем квадратичный невычет по модулю $p$).
%\item С помощью деления на 2, найдем такие $s$ и $t$, $p - 1 = 2^{s}t$, где
%$t$ -- нечетно.
%\item Вычислим $a^{-1}\bmod{p}$
%\begin{enumerate}
%\item При помощи расширенного алгоритма Эвклида найдем $x$ и $y$ такие что $ax + py = d$, где $d = \textrm{НОД}(a, p)$.
%\item Если $d > 1$, тогда $a^{-1}\bmod{p}$ не существует. Иначе, оно равно $x$.
%\end{enumerate}
%\item Установим $c \leftarrow b^t \bmod{p}$ и $r \leftarrow a^{(t+1)/2} \bmod{p}$
%\item Для $i$ от 1 до $s - 1$, будем выполнять следующее:
%\begin{enumerate}
%\item Вычислим $d = (r^2 \cdot a^{-1})^{2^{s - i -1}} \bmod{p}$.
%\item Если $d \equiv -1 \imod{p}$, то установим $r \leftarrow r \cdot c \bmod{p}$.
%\item Установим $c \leftarrow c^2 \bmod{p}$.
%\end{enumerate}
%\item Ответ: $r$ и $-r$.
%\end{enumerate}

%Этот алгоритм зависит от случайного выбора из за способа поиска квадратичного
%невычета. Даже не смотря на то что половина элементов $Z^*_p$ являются
%квадратичными невычетами по модулю $p$, не известно ни одного 
%детерминированного алгоритма их поиска, работающего за полиномиальное время.
%Ожидаемое время работы при случайном переборе --- 2 итерации, поэтому
%можно считать что процедура занимает полиномиальное время.


\chapter{Используемые протоколы}

%\section{Протокол подбрасывания монетки с помощью квадратных
%    корней}

%Можно разделить этот протокол на две части --- собственно
%протокол бросания монеты, который определяет результат,
%и протокол проверки, который необходим для того чтобы
%удостовериться в том что участники взаимодействия играли
%честно.\\

%Подпротокол бросания монеты:

%\begin{enumerate}
%\item Алиса выбирает два больших простых числа, $p$
%и $q$, и посылает их произведение $n$ Бобу.
%\item Боб выбирает случайное положительное целое
%число $r$, меньшее $n/2$. Боб вычисляет
%$$z = r^2 \bmod{n}$$ и посылает $z$ Алисе.
%\item Алиса вычисляет четыре квадратных корня $z \imod{n}$.
%Она может сделать это, так как она знает разложение $n$ на множители
%(как описано в методе \ref{method1}). Назовем их $+x$, $-x$, $+y$ и
%$-y$. Обозначим как $x'$ меньшее из следующих двух чисел:
%$$x \bmod{n}$$
%$$-x \bmod{n}$$
%Аналогично, обозначим как $y'$ меньшее из следующих двух чисел:
%$$y \bmod{n}$$
%$$-y \bmod{n}$$
%Обратите внимание, что $r$ равно либо $x'$, либо $y'$.
%\item \label{lbl1} Алиса делает пытается угадать, какое из значений,
%$x'$ или $y'$ равно $r$, и посылает свою догадку Бобу.
%\item Если догадка Алисы правильна, результатом броска
%монеты является <<орел>>, а если неправильна --- <<решка>>.
%Боб объявляет результат броска монеты.
%\newcounter{enumi_saved}
%\setcounter{enumi_saved}{\value{enumi}}
%\end{enumerate}
%
%Подпротокол проверки:
%
%\begin{enumerate}
%\setcounter{enumi}{\value{enumi_saved}}
%\item Алиса посылает $p$ и $q$ Бобу.
%\item Боб вычисляет $x'$ и $y'$ и посылает их Алисе.
%\item Алиса вычисляет $r$.
%
%\end{enumerate}
%
%У Алисы нет возможности узнать $r$, поэтому она действительно
%угадывает. На этапе \ref{lbl1} она сообщает Бобу только один бит
%своей догадки, не давая ему получить и $x'$, и $y'$. Если Боб
%получит оба этих числа, он сможет изменить $r$ после 
%этапа \ref{lbl1}.


\section{Протокол доказательства с
нулевым разглашением знания закрытого ключа RSA}

Алиса знает закрытый ключ Кэрол. Может быть она взломала RSA, а
может она взломала дверь квартиры Кэрол и выкрала ключ. Алиса
хочет убедить Боба, что ей известен ключ Кэрол. Однако она не
хочет ни сообщать Бобу ключ, ни даже расшифровать для Боба одно
из сообщений Кэрол. Далее приведен протокол с нулевым знанием,
с помощью которого Алиса убеждает Боба, что она знает закрытый
ключ Кэрол.\\

Пусть открытый ключ Кэрол --- $e$, ее закрытый ключ --- $d$, а
модуль RSA --- $n$.

\begin{enumerate}
\item Алиса и Боб выбирают случайное $k$ и $m$, 
для которых $$km \equiv e \imod{\varphi(n)}$$ Числа они 
должны выбирать случайным образом, используя для
генерации $k$ протокол бросания монеты, а затем 
вычисляя $m$. Если и $k$, и $m$ больше $3$,
протокол продолжается. В противном случае числа
выбираются заново.
\item Алиса и Боб генерируют случайный шифротекст
$C$. И снова они должны воспользоваться протоколом
бросания монеты.
\item Алиса, используя закрытый ключ Кэрол, вычисляет
$$M = C^d\bmod{n}$$ Затем она вычисляет 
$$ X = M^k\bmod{n}$$ и посылает $X$ Бобу.
\item Боб проверяет, что $$X^m\bmod{n} = C$$
Если это так, то он убеждается в правильности 
заявления Алисы.
\end{enumerate}

Аналогичный протокол можно использовать для демонстрации
возможности вскрытия проблемы дискретного логарифма.


\chapter{Реализация}


\chapter*{Используемая литература}

\begin{enumerate}
\item Handbook of Applied Cryptography -- Alfred J. Menezes, Paul C. van Oorschot, Scott A. Vanstone
\item Applied Cryptography -- Bruce Schneier
\end{enumerate}

\end{document}
